\Introduction

В современном мире всё более и более востребованным становится использование компьютерной графики в разных сферах деятельности. Она используется в компьютерном моделировании, компьютерных играх, а также в системах автоматизированного проектирования. В связи с нарастающей потребностью к реалистичному изображению трехмерных объектов, появляется необходимость создавать программное обеспечение, способное учитывать широкий спектр оптических явлений при отрисовке изображения.

В компьютерной графике алгоритмы создания реалистичного изображения требуют особых затрат ресурсов, тратится много времени и памяти для получения результата. Затраты времени на синтез изображения не всегда позволяют удобно и эффективно подобрать параметры отрисовки.

Цель данной работы – создать программное обеспечение, которое позволило бы синтезировать изображение трехмерной сцены с настраиваемыми параметрами ее объектов в реальном времени и сменой режимов быстродействия.

Для достижения поставленной цели, необходимо решить следующие задачи:

\begin{itemize}
	\item описать трехмерную сцену и объекты, из которых она состоит;
	\item анализ и видоизменение существующих алгоритмов отрисовки, визуализирующих трехмерную сцену;
	\item реализовать выбранные алгоритмы;
	\item разработать программное обеспечение, позволяющее в реальном времени менять параметры сцены, а также параметры отрисовки.
\end{itemize}
