\chapter*{Введение}
\addcontentsline{toc}{chapter}{Введение}

В современном мире всё более и более востребованным становится использование компьютерной графики в разных сферах деятельности. Машинная графика используется в компьютерном моделировании, компьютерных играх и системах автоматизированного проектирования \cite{cad}. В связи с нарастающей потребностью к реалистичному изображению трехмерных объектов, появляется необходимость создавать программное обеспечение, способное учитывать широкий спектр оптических явлений при отрисовке изображения \cite{optics}.

В компьютерной графике алгоритмы создания реалистичного изображения требуют особых затрат ресурсов, тратится много времени и памяти для получения результата. Затраты времени на синтез изображения не всегда позволяют удобно и эффективно подобрать параметры отрисовки.

Целью данной работы является создание программного обеспечения, которое позволило бы синтезировать изображение трехмерной сцены с настраиваемыми параметрами ее объектов в реальном времени и сменой режимов быстродействия.

Для достижения поставленной цели необходимо решить следующие задачи:

\begin{itemize}
    \item исследовать подходы к синтезу реалистичных изображений;
    \item сравнить существующие решения;
    \item описать структура разрабатываемого ПО;
    \item описать используемые при разработке ПО алгоритмы;
    \item определить средства программной реализации;
    \item реализовать алгоритмы отрисовки сцены;
    \item описать процесс сборки приложения;
    \item протестировать разработанное ПО;
    \item сравнить производительность реализованных алгоритмах на основе разных сцен.
\end{itemize}