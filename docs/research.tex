\chapter{Исследовательский раздел}

В данном разделе представлены эксперименты для проведения сравнительного анализа быстродействия реализуемых алгоритмов синтеза изображения.

\section{Постановка эксперимента}

В эксперименте исследуется зависимость времени выполнения алгоритмов синтеза изобржения в зависимости от количества объектов сцены при двух и пяти различных источниках освещения.

При проведении эксперимента все объекты сцены будут расположены в поле зрения камеры, чтобы исключить случаи, когда реальная отрисовка объекта не выполняется.

Количество объектов сцены будет варьироваться от 1 по 7 с шагом 2.

Для каждого случая будет рассчитано время синтеза изображения алгоритмом предобработки (быстрая отрисовка) и алгоритмом трассировки лучей.

Полученный результат не усредняется.

\section{Технические характеристики}

Ниже приведены характеристики устройства, на котором будет производиться эксперимент:

\begin{itemize}
    \item операционная система: Pop!\_OS 21.04 \cite{pop} Linux 64-bit;
    \item оперативная память: 8Gb;
    \item процессор: Intel Core i7-8750H @ 12x 4.1 GHz.
\end{itemize}

\section{Результаты эксперимента}

Ниже в таблице \ref{exp:res} представлены результаты эксперимента.

\begin{table}[h]
	\caption{Время работы алгоритмов при различных параметрах сцены}
	\begin{center}
		\begin{tabular}{|c|c|c|c|}
			\hline
			\specialcell{кол-во объектов \\ в сцене} &
			\specialcell{кол-во источников \\ освещения} &
			\specialcell{время \\ закраски} &
			\specialcell{время \\ трассировки} \\
			\hline
			1 & 2 & 10.3 мс & 10.5 с \\
			\hline
			3 & 2 & 12.2 мс & 25.7 с \\
			\hline
			5 & 2 & 15.4 мс & 107 с \\
			\hline
			7 & 2 & 31.5 мс & 174 с \\
			\hline
			1 & 5 & 10.3 мс & 14.8 с \\
			\hline
			3 & 5 & 13.7 мс & 62.1 с \\
			\hline
			5 & 5 & 15.9 мс & 334 с \\
			\hline
			7 & 5 & 35.8 мс & 1520 с \\
			\hline
		\end{tabular}
	\end{center}
	\label{exp:res}
\end{table}

Как видно из таблицы \ref{exp:res}, время затрачиваемое на отрисовку с использованием трассировки лучей в тысячу раз превосходит время затрачиваемое быстрым алгоритмом.

Также можно заметить, что количество источников освещения незначительно влияет на время при быстрой закраске, однако на порядок увеличивает время синтеза изображения методом трассировки лучей.
