\chapter*{Заключение}
\addcontentsline{toc}{chapter}{Заключение}

В результате выполнения курсовой работы были исследованы основные подходы к синтезу реалистичных изображений, описана структура разрабатываемого программного обеспечения, а также описаны разрабатываемые алгоритмы и структуры данных. Был разработан программный продукт, целью которого является возможность интерактивной смены параметров сцены для последующего синтеза реалистичного изображения с помощью трассировки лучей.

Программа реализована таким образом, что пользователь может добавлять и изменять объекты, редактировать освещение, изменять положение камер в режиме реального времени, наблюдая при этом результат быстрого менее реалистичного синтеза изображения. Пользователь может выбирать между различными алгоритмами отрисовки и менять метод освещения.

В ходе выполнения исследовательской части было установлено, что интерактивный режим рендеринга мало зависит от количества источников освещения и синтезирует изображение в среднем в 10000 раз быстрее (при возрастании количества объектов нелинейно возрастает время, затрачиваемое на синтез изображения).