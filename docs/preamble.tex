\usepackage[T2A]{fontenc}
\usepackage[utf8]{inputenc}
\usepackage[english,russian]{babel}

\usepackage[left=30mm, right=20mm, top=20mm, bottom=20mm]{geometry}

\usepackage{microtype}
\sloppy

\usepackage{setspace}
\onehalfspacing

\usepackage{indentfirst}
\setlength{\parindent}{12.5mm}

\usepackage{titlesec}
\titleformat{\chapter}{\LARGE\bfseries}{\thechapter}{20pt}{\LARGE\bfseries}
\titleformat{\section}{\Large\bfseries}{\thesection}{20pt}{\Large\bfseries}

\addto{\captionsrussian}{\renewcommand*{\contentsname}{Содержание}}
\usepackage{natbib}
\renewcommand{\bibsection}{\chapter*{Список литературы}}

\usepackage{caption}

\usepackage{wrapfig}
\usepackage{float}

\usepackage{graphicx}
\newcommand{\imgwc}[4]
{
	\begin{figure}[#1]
		\center{\includegraphics[width=#2]{inc/img/#3}}
		\caption{#4}
		\label{img:#3}
	\end{figure}
}
\newcommand{\imghc}[4]
{
	\begin{figure}[#1]
		\center{\includegraphics[height=#2]{inc/img/#3}}
		\caption{#4}
		\label{img:#3}
	\end{figure}
}
\newcommand{\imgsc}[4]
{
	\begin{figure}[#1]
		\center{\includegraphics[scale=#2]{inc/img/#3}}
		\caption{#4}
		\label{img:#3}
	\end{figure}
}

\newcommand{\specialcell}[2][c]
{
	\begin{tabular}[#1]
		{@{}c@{}}#2
	\end{tabular}
}

\usepackage{pgfplots}
\pgfplotsset{compat=newest}

\usepackage{listings}
\usepackage{listingsutf8}
\lstset{
	basicstyle=\footnotesize\ttfamily,
	keywordstyle=\color{blue},
	stringstyle=\color{red},
	commentstyle=\color{gray},
	numbers=left,
	numberstyle=\tiny,
	numbersep=5pt,
	frame=false,
	breaklines=true,
	breakatwhitespace=true,
	inputencoding=utf8/koi8-r,
	columns=fullflexible
}

\newcommand{\code}[1]{\texttt{#1}}

\usepackage{amsmath}
\usepackage{amssymb}

\usepackage[hyphens]{url}
\usepackage[unicode]{hyperref}
\hypersetup{hidelinks}
\hypersetup{breaklinks=true}

\makeatletter
\newcommand{\vhrulefill}[1]
{
	\leavevmode\leaders\hrule\@height#1\hfill \kern\z@
}
\makeatother