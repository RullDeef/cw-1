\chapter{Технологический раздел}

\section{Средства реализации}

Для реализации ПО был выбран язык программирования C++ \cite{c++} по причине его эффективности, прямой работой с памятью и широкого распространения в области компьютерной графики.

\section{Листинги реализаций алгоритмов}

\subsection{Алгоритм трассировки лучей}

\begin{lstlisting}[language=C++]
static std::pair<double, Color> traceRay(Ray ray, const Scene& scene, const Camera& camera, int depthLeft) {
  Mesh* mesh;
  Face* face;
  double distance;
  
  if (depthLeft > 0 && findIntersectionPoint(mesh, face, distance, ray, scene)) {
    Vec position = ray_at(ray, distance);
    Vec view = normalized(position - camera.eye);
    Mat view_mat = inverse(camera.model_mat);
    Vec normal = get_normal_at(mesh->model_mat * *face, position);
    
    Color colorReflected = Colors::black;
    Color colorTransmitted = Colors::black;
    Color colorLighted = Colors::black;
    
    if (mesh->material.coefReflection > 0) {
      Vec reflected = ray.direction - 2 * dot(ray.direction, normal) * normal;
      auto traced = traceRay(make_ray(position + 0.01 * reflected, reflected), scene, camera, depthLeft - 1);
      if (traced.first < std::numeric_limits<double>::infinity())
        colorReflected += mesh->material.coefReflection * traced.second;
    }
    if (mesh->material.coefRefraction > 0) {
      double d = dot(-ray.direction, normal);
      if (d > 0) {
        double theta = std::acos(d);
        Vec m = (normal * std::cos(theta) + ray.direction) / std::sin(theta);
        Vec refracted = normalized(m * std::sin(theta) - normal * std::cos(theta));
        auto traced = traceRay(make_ray(position + 0.01 * refracted, refracted), scene, camera, depthLeft - 1);
        if (traced.first < std::numeric_limits<double>::infinity())
          colorTransmitted += mesh->material.coefRefraction * traced.second;
      }
    }
    
    for (size_t i = 0; i < scene.lightList.size; i++) {
      Light light;
  
      if (!get(scene.lightList, i, light))
        return {std::numeric_limits<double>::infinity(), Colors::black};
      else if (!light.visible)
        continue;
      else {
        if (light.type == LightType::Ambient) {
          colorLighted += light.intensity * light.color * mesh->material.ambientColor;
        }
        else if (light.type == LightType::Directional) {
          if (dot(normal, -light.direction) > 0) {
            auto traced = traceRay(make_ray(position - 0.1 * light.direction, -light.direction), scene, camera, depthLeft - 1);
            if (traced.first == std::numeric_limits<double>::infinity()) {
              colorLighted += dot(normal, -light.direction) * light.intensity * light.color;
            }
          }
        }
        else if (light.type == LightType::Point) {
          auto traced = traceRay(make_ray(position, normalized(light.position - position)), scene, camera, depthLeft - 1);
          colorLighted += traced.second;
        }
      }
    }
    
    double op = mesh->material.opacity;
    Color color = op * colorLighted + (1 - op) * colorTransmitted + colorReflected;
  
    return {distance, color};
  }
  
  Color color = Colors::black;
  
  for (size_t i = 0; i < scene.lightList.size; i++) {
    Light light{};
    get(scene.lightList, i, light);
    
    if (!light.visible)
      continue;
    else if (light.type == LightType::Ambient)
      color += light.intensity * light.color;
    else if (light.type == LightType::Directional) {
      double val = dot(ray.direction, -light.direction);
      if (val > 0)
        color += val * light.intensity * light.color;
    }
  }
  
  return {std::numeric_limits<double>::infinity(), color};
}
\end{lstlisting}

\subsection{Алгоритм сплошной закраски полигона}

\begin{lstlisting}[language=C++]
void Core::renderFlat(RenderTarget& renderTarget, ZBuffer& zbuffer, RenderRegion region, ColorComputeFn colorComputeFn) {
  double xLeft = region.xStartLeft;
  double xRight = region.xStartRight;
	
  double zLeft = region.zStartLeft;
  double zRight = region.zStartRight;
	
  Pixel color = to_pixel(colorComputeFn(region.center, region.normal, region.meshPtr->material));
	
  for (int y = region.yStart; y <= region.yEnd; y++) {
    double z = zLeft;
    double dz = (zRight - zLeft) / (xRight - xLeft + 1);
		
    if (0 <= y && y < renderTarget.height) {
      for (auto x = (int)xLeft; x < xRight; x++) {
        if (0 <= x && x < renderTarget.width)
          updatePixel(renderTarget, zbuffer, y, x, 1.0 / z, color);

        z += dz;
      }
    }

    xLeft += region.dxLeft;
    xRight += region.dxRight;

    zLeft += region.dzLeft;
    zRight += region.dzRight;
  }
}
\end{lstlisting}
